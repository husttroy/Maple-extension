\section{Summary}
\label{sec:summary}
The main contribution of this paper is the design and implemenation of the Chrome extension of {\tool}, which provides browser-based tool support for systematically assessing and augmenting Stack Overflow with common API usage patterns mined from GitHub. In our previous work~\cite{zhang2018code}, we examined 217K SO posts with 180 validated patterns and found that 31\% of SO posts might have potential API usage violations. The first two authors manually inspected 400 SO posts with detected API usage violations and confirmed real API misuse in 289 posts, which could produce symptoms such as program crashes and resource leak if reused {\em as-is} to a target project. We also found that many unreliable examples were simplified to operate on crafted input data for illustration purposes only. Such curated examples are not sufficient for various input data and usage scenarios in real software systems, especially for handling corner cases. Both the dataset of API usage patterns and the SO posts with potential API usage violations are publicly available for download.\footnote{\url{http://web.cs.ucla.edu/~tianyi.zhang/examplecheck.html}} The Chrome extension of {\tool} is also published in Chrome Web Store~\cite{examplecheck}.

%This paper introduces a Chrome extension, {\tool} that proactively detects API usage violations in a Stack Overflow post and enriches the post with extra API usage tips evidenced by a large number of GitHub code examples. Certainly, SO snippets are supposed to provide a starting point, not necessarily being fully complete or reliable. However, such incomplete or unreliable snippets can potentially impact the production code, when a programmer mentally or physically reuses a snippet to a target project. {\tool} can help programmers implicitly assess a given SO snippet and reduce the effort of cross-checking and testing the snippet when reusing it to a target project. 

As future work, we plan to conduct a longitudinal study with Stack Overflow users to understand the adoption and usage of {\tool}. We are in the process of intrumenting {\tool} to log user interaction behavior. By analyzing user behavior and conducting post surveys, we can gain both quantitative and qualitative insights on the usage of {\tool} as well as improvement opportunities. %We also want to design code completion tasks and conduct a controlled lab study to evaluate whether using {\tool} indeed helps participants write more reliable code.
