% ==================================================
% FSE demo paper
%   4 pages in the ACM proceedings format, including all text, references and figures.
%
%   The paper must communicate clearly the following information to the audience:
%      1. the envisioned users
%      2. the software engineering challenge it proposes to address;
%      3. the methodology it implies for its users
%      4. the results of validation studies already conducted for mature tools, or the design of planned studies for early prototypes.
%   The paper must be accompanied by a short video (between 3 and 5 minutes long)
%
%   Paper    submission : June 29, 2018
%   Author   notification : July 16, 2018
%   Camera-ready deadline : July 30, 2018
% ==================================================
\documentclass[sigconf, review]{acmart}
%\pdfpagewidth=8.5in
%\pdfpageheight=11in
\input{macros}

\setcopyright{rightsretained}
\acmConference[ESEC/FSE 2018]{The 26th ACM Joint European Software Engineering Conference and Symposium on the Foundations of Software Engineering}{4--9 November, 2018}{Lake Buena Vista, Florida, United States}

% correct bad hyphenation here
\hyphenation{op-tical net-works semi-conduc-tor}

\begin{document}

\title{Augmenting Stack Overflow with API Usage Patterns Mined from GitHub}

\author{Anastasia Reinhardt\textsuperscript{1}\footnotemark\,\,Tianyi Zhang\textsuperscript{2}\,\,Mihir Mathur\textsuperscript{2}\,\,Miryung Kim\textsuperscript{2}}
\authornote{Work done as an intern at University of California, Los Angeles.}
\affiliation{\textsuperscript{1}George Fox University}
\affiliation{\textsuperscript{2}University of California, Los Angeles}
\affiliation{areinhardt14@georgefox.edu, \{tianyi.zhang, miryung\}@cs.ucla.edu, mihirmathur@ucla.edu}

\renewcommand{\authors}{Anastasia Reinhart, Tianyi Zhang, Mihir Mathur, and Miryung Kim}
\renewcommand{\shortauthors}{Anastasia Reinhart, Tianyi Zhang, Mihir Mathur, and Miryung Kim}


%\author{Anastasia Reinhardt}
%\affiliation{George Fox University}
%\email{areinhardt14@georgefox.edu}
%
%\author{Tianyi Zhang, Mihir Mathur, Miryung Kim}
%\affiliation{University of California, Los Angeles}
%\email{tianyi.zhang@cs.ucla.edu}
%
%\author{Mihir Mathur}
%\affiliation{University of California, Los Angeles}
%\email{mihirmathur@ucla.edu}
%
%\author{Miryung Kim}
%\affiliation{University of California, Los Angeles}
%\email{miryung@cs.ucla.edu}


% ==================================================
% abstract
% ==================================================
\begin{abstract}
Programmers often consult Q\&A websites such as Stack Overflow to learn new APIs. However, online code snippets are not always complete or reliable in terms of API usage. To help programmers assess online code snippets, we build a Chrome-extension, {\tool} that detects API usage violations in Stack Overflow posts using patterns mined from 380K GitHub projects. {\tool} quantifies how many GitHub examples follow the correct API usage and also provides the remediation of a detected violation. With the assistance of {\tool}, programmers can easily identify the limitations or pitfalls in a Stack Overflow snippet without cross-checking multiple posts for proper API usage reference. They can also build confidence on a code snippet by learning how much it is aligned with other similar code in GitHub. {\tool} is available at Chrome Web Store\footnote{Download the Chrome extension of {\tool} at \url{https://chrome.google.com/webstore/detail/examplecheck/amliempebckaiaklimcpopomlnklkioe}} and the demo video is available at~\todo{add the youtube video link here}.
\end{abstract}


\begin{CCSXML}
<ccs2012>
<concept>
<concept_id>10011007.10010940.10011003.10011004</concept_id>
<concept_desc>Software and its engineering~Software reliability</concept_desc>
<concept_significance>300</concept_significance>
</concept>
<concept>
<concept_id>10011007.10011074.10011134</concept_id>
<concept_desc>Software and its engineering~Collaboration in software development</concept_desc>
<concept_significance>300</concept_significance>
</concept>
<concept>
<concept_id>10011007.10011006.10011066.10011069</concept_id>
<concept_desc>Software and its engineering~Integrated and visual development environments</concept_desc>
<concept_significance>300</concept_significance>
</concept>
</ccs2012>
\end{CCSXML}

\ccsdesc[300]{Software and its engineering~Software reliability}
\ccsdesc[300]{Software and its engineering~Collaboration in software development}
\ccsdesc[300]{Software and its engineering~Integrated and visual development environments}

\keywords{online Q\&A forum, API usage pattern, code assessment}  

\maketitle

% ==================================================
% Introduction
% ==================================================
% ==============================================
\section{Introduction}
\label{sec:intro}
% ==============================================
% Problem
Programmers often search for online code examples to learn new APIs. A case study at Google shows that developers issue an average of 12 code search queries per weekday~\cite{sadowski2015developers}. Stack Overflow (SO) is a popular Q\&A website that programmers often consult. In July 2017, Stack Overflow has accumulated more than 22 million answers, many of which contain code examples to demonstrate the solution for a particular programming question. However, SO examples are not always complete or reliable, which can be misleading and potentially dangerous when programmers follow the same example to complete a client program. For example, Fischer et al.~found that 29\% of security-related code snippets on Stack Overflow were insecure and have potentially affected over 1 million Android apps on Google play~\cite{fischer2017stack}. 

% Solution
This paper presents {\tool}, a Chrome extension that augments Stack Overflow with common API usage patterns learned from GitHub and alerts programmers about the potential violations in a Stack Overflow post. Current {\tool} includes hundreds of API usage patterns learned from 380K GitHub repositories.\footnote{The entire dataset of patterns is uploaded with EasyChair.} These patterns represent three types of API-related usage---temporal ordering, guard conditions, and exception handling logic of API methods. Our insight is that commonly practiced idioms in massive code corpora may represent a desirable pattern that a programmer can use to trust and enhance code examples on Stack Overflow. 

Given a Stack Overflow post, {\tool} first extracts the sequence of API calls with corresponding control constructs and guard conditions. {\tool} then contrasts the sequence with the dataset of API usage patterns learned from GitHub and highlights method calls that violate the common usage patterns. To help users better understand the violations, {\tool} further generates descriptive warning messages and also suggests a corrected usage example with the same variable names as in the Stack Overflow code snippet. Mining API usage patterns to detect violations often suffers from reporting false alarms, since mined patterns may not be inclusive and fit all use scenarios of an API~\cite{liang2016antminer}. To mitigate this issue, {\tool} allows users to upvote or downvote a violation based on its applicability and usefulness to a Stack Overflow post. {\tool} filters a pattern when multiple users flag it unhelpful to constrast against a given snippet. To help developers build confidence on a detected violation, {\tool} shows how many GitHub developers also follow the pattern as well as how many other users like or dislike this pattern.

\begin{figure}
\centering
\includegraphics[width=0.5\textwidth]{soap-v3.pdf}
  \vspace{.1in}
  \caption{{\tool} Chrome extension that augments Stack Overflow with API misuse warning. The pop-up window alerts that {\ttt match\_number} can be {\ttt null} if the requested {\ttt JSON} attribute does not exist and will crash the program by throwing {\ttt NullPointerException} when {\ttt getAsString} is called on it.\protect\footnotemark}
  \label{fig:screenshot}
\end{figure}

\footnotetext{\url{https://stackoverflow.com/questions/29860000}}

A user of {\tool} would benefit from the addition of concrete examples from the production code in GitHub  to the code examples she encounters on Stack Overflow. This will not only combat programming issues stemming from the use of incomplete or unreliable Stack Overflow code examples, but will also be an aid for users learning a new API. By enhancing examples already found in Stack Overflow, a user can trust that the shown example follows a common and reliable usage pattern for a given API method.
%A user of this tool would benefit from not needing to cross-reference multiple sites for proper API usage reference, and will be able to continue using Stack Overflow to learn APIs with the added advantage of seeing which usage patterns a post may have left out of its explanation. This could result in more complete, reliable code with minimal added time or effort on the part of the programmer.\todo{The description here does not sound very appealing. Can you rework this paragraph?} 

The main contribution of this paper is to describe the features of {\tool} from a user's perspective. In order to give programmers access to {\tool}, the front-end of {\tool} is implemented as a Chrome extension that users can easily download and install.\todo{add a url to download our tool} The detailed algorithm and study are described in our separate technical report.\todo{cite Maple}

%input{fig_motiExample}
%\input{fig_UI}


% ==================================================
% Motivating Example and Tool Features
% ==================================================
% ==============================================
% !TEX root = ./critics.tex
% ==============================================
% ==============================================
\section{Motivating Examples and Tool Features}
\label{sec:motivation}
% ==============================================
\begin{figure}
\includegraphics[width=0.48\textwidth]{json_ex1.PNG}
\vspace{.1in}
\caption{A code snippet that does not properly check {\tt JsonElement.getAsString}.\protect\footnotemark}
\label{fig:arch}
\end{figure}

\footnotetext{https://stackoverflow.com/questions/34120882/gson-jsonelement-getasstring-vs-jsonelement-tostring}

Consider Alice, a software developer who needs to develop a feature for a program using Google's Gson library for Java\footnote{https://github.com/google/gson/blob/master/UserGuide.md}, which she is unfamiliar with. Alice uses an Internet search engine to look up how to use Gson's {\tt JsonElement}'s {\tt getAsString()} method and finds a post on Stack Overflow, as shown in Figure 2. However, this example does not use the {\tt JsonElement} API completely correctly. 

{\bf Stack Overflow Post View.}
Alice is running Maple's Chrome extension when she finds this post, and the extension highlights the potential API misuse in the code snippet, as seen in Figure 2. Note that the extension only highlights the first instance of this misuse although it occurs multiple times in the code snippet, to eliminate redundancy and avoid confusion. Alice is interested in learning more about the API and what specifically the code snippet did not include, so she clicks on the highlighted text.

\begin{figure}
\centering
  \begin{subfigure}[a]{0.48\textwidth}
  \includegraphics[width=\textwidth]{json_ex2.PNG}
  \caption{A page describing a way to avoid a {\tt ClassCastException} by checking whether the {\tt JsonElement} object is a primitive.} 
  \vspace{.1in}
  \label{fig:arch}
  \end{subfigure}
  \hfill
  \begin{subfigure}[b]{0.48\textwidth}
  \includegraphics[width=\textwidth]{json_ex3.PNG}
  \caption{A page describing a way to avoid a {\tt NullPointerException} by checking whether the {\tt JsonElement} object is null.}
  \vspace{.1in}
  \label{fig:arch}
  \end{subfigure}
  \hfill
\caption{The two pages of a popup generated on {\tt JsonElement.getAsString}.}
\end{figure}

{\bf Stack Overflow Popup View.}
Clicking on the highlighted text reveals a popup, as seen in Figure 3. The popup is populated with information about any required patterns in Maple's database this particular API call does not adhere to. Alice notices that there are two pages of the popup, indicating two different usage patterns that this call does not follow, as shown in Figures 3a and 3b. 

Alice inspects the first page (Fig. 3a) and learns that she should check whether the {\tt JsonElement} object is a primitive before calling {\tt getAsString} in order to avoid a {\tt ClassCastException}. She notices that 52 other GitHub code examples use this pattern, which gives her a qualitative measurement of how prevalent this pattern is in compiled, "real world" code.
Alice then inspects the second page (Fig. 3b), and finds that it suggests a null check before calling {\tt getAsString}. She notices that this pattern has more than double the support of the previous pattern.

Curious to see the first pattern in context, Alice returns to the first page of the popup and clicks on the first link provided to her under "See this in a GitHub example."

\begin{figure}
\centering
  \begin{subfigure}[a]{0.48\textwidth}
  \includegraphics[width=\textwidth]{json_primitive_gh1.PNG}
  \caption{The first GitHub example for Figure 3a.} 
  \vspace{.1in}
  \label{fig:arch}
  \end{subfigure}
  \hfill
  \begin{subfigure}[b]{0.48\textwidth}
  \includegraphics[width=\textwidth]{json_null_gh2.PNG}
  \caption{The second GitHub example for Figure 3b.}
  \vspace{.1in}
  \label{fig:arch}
  \end{subfigure}
  \hfill
\caption{The highlighted GitHub examples redirected to from the links provided in the popup.}
\end{figure}

{\bf GitHub Example View.} When Alice clicks on one of the GitHub links, the file opens in a new tab and the view scrolls to where the API is called in the file, and the method in which this occurs is highlighted so Alice can easily find it, as seen in Figure 4. The addition of a compilable code example that demonstrates the pattern in context can aid Alice in understanding how to use the pattern if it is unfamiliar to her. In this case, Alice finds herself redirected the method in a GitHub project seen in Figure 4a. She notices that the example uses a null check in conjunction with the primitive check, which makes sense to her after seeing that both were missing from the Stack Overflow code snippet.

Returning to the popup in Stack Overflow, Alice clicks on the second link provided for the second page to compare usage patterns in context. This link opens up to the GitHub method seen in Figure 4b. Unlike the one in Figure 4a, this example does not use the null check in conjunction with the primitive check. 

After seeing these two examples, Alice can infer that a null check is more necessary and is more common than the primitive check, based on the GitHub examples she has seen as well as the GitHub support indicated by the popup message. She upvotes the null check's pattern by clicking on the up-arrow on its page (see Figure 3b) to send the server her feedback on the patterns it gave her. 



% ==================================================
% Approach
% ==================================================
% ==============================================
% !TEX root = ./critics.tex
% ==============================================
% ==============================================
\section{Implementation}
\label{sec:implementation}
% ==============================================
This section describes the implementation details of the Maple plugin. It is implemented as an Chrome extension and consists of five components: (1) Stack Overflow code snippet extraction, (2) API misuse detection, (3) popup generation, (4) GitHub example alteration, and (5) a user feedback interface.

{\bf Code Snippet Extraction.} When a user loads a page in the Stack Overflow domain, the plug-in side of the Maple interface extracts any text in answer posts that appears within {\tt <code>} tags, and sends them in a JSON message to the server, as seen in Figure 1.

{\bf API Misuse Detection.} When the server receives a message from the plug-in, it parses the snippets into API call sequences, abstracting away irrelevant statements and syntactic details. An API call sequence consists of relevant control constructs and API calls, where the API calls are annotated with the number of arguments as well as any guard conditions associated with it.

Once the code snippets are parsed, Maple searches a MySQL database for the API calls present in each API call sequence. It receives the required and alternative patterns associated with these calls, and checks whether the code snippet's call sequence satisfies one of the alternative patterns and all required patterns. A code snippet's call sequence satisfies a pattern if it is subsumed by it.

Additionally, while this checking process is occurring, the guard conditions in the code snippets are generalized before checking their logical implications using the SMT solver Z3. Z3 is used to check the logical equivalence between two guard conditions so that Maple can merge logically-similar clusters of guard conditions, making it able to prove the semantic equivalence of conditions like {\tt arg0<arg1} and {\tt arg1>arg0} regardless of their syntactic similarity [much of this is lifted pretty directly from p4 of the Maple paper...].

If Maple finds that an API call sequence does not satisfy the necessary patterns, it generates violations for each potential API misuse. The server collects these violations, along with the required pattern being violated, and maps them to the API call that generated the violation. This data is wrapped into a single JSON message and returned to the plug-in.

{\bf Popup Generation.} Using the data from the server's JSON message, the plug-in searches the identified code snippet for the API call in question, highlights it, and generates a Bootstrap popover on it, as seen in Figure 3. The popover is populated with a violation message describing the pattern being violated and including the GitHub support for the required pattern, in terms of number of supporting projects as well as three links to relevant GitHub pages using that pattern correctly. The plug-in generates one popover for each API call, and generates pages of the popover for each different pattern being violated by that call. The provided example of the pattern is currently hard-coded.\todo{We need to provide more details about how to generate the text description of each violation, e.g., showing the templates in a table.}

{\bf GitHub Example Alteration.} When the user clicks on a provided GitHub example in the popup, the plug-in's main script writes the name of the method call associated with that link to a shared storage space for the Chrome extension, using the chrome.storage API. When the GitHub page is loaded, the plug-in's secondary script is activated and checks the storage for any messages. If a method name has been written there, the secondary script searches the GitHub page for a method declaration of the same name. This is not necessarily the name of the API call the popup has been generated on; this is a method in a GitHub code file that uses that API call in a pattern that the Stack Overflow code snippet does not adhere to. The script highlights the entire method and scrolls the view down so the user is easily able to find it, as seen in Figure 4.

{\bf User Feedback.} Users are able to give feedback on the patterns the popup shows them by voting "up" or "down" on them. When a vote is registered, the plug-in sends the server a message with the pattern's specific ID and a vote of +1 or -1, which the server uses to update the database. This will be used in the future to rank patterns that are sent to the plug-in, and learn which ones users find helpful or unhelpful.

% ==================================================
% Related Work
% ==================================================
\input{related}

% ==================================================
% Summary
% ==================================================
\section{Summary}
\label{sec:summary}
Programmers often resort to Stack Overflow to learn about new APIs. Certainly, code snippets on Stack Overflow are supposed to provide a starting point, not necessarily being fully complete or reliable. On the other hand, such incomplete or unreliable snippets can potentially impact the production code, when a programmer mentally or physically reuses a snippet to a target project. 
%Recognizing unreliable snippets requires programmers to cross-check multiple code snippets until finding an ideal usage or extensively test for corner cases, which is difficult due to limited time and attention. 
To bridge the gap, this paper presents a Chrome extension, {\tool} that proactively detects potential API usage violations in a Stack Overflow post and enriches the post with extra tips about correct API usage evidenced by a large number of GitHub snippets. {\tool} could help programmers to implicitly assess a given code snippet on Stack Overflow and reduce the effort of cross-checking and testing the snippet when reusing it to a target project. 

{\bf\em Future work.} We plan to conduct a longitudinal analysis as well as a lab study to evaluate the usefulness of {\tool}. We have instrumented our tool and published it in Chrome Web Store. By collecting the user traces of how they collect aodw 

%\section{Acknowledgement} 

\balance
\bibliographystyle{ACM-Reference-Format}
\bibliography{examplecheck}
\end{document}
